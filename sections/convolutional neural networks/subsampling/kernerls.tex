The subsampling CNNs perform is not related specifically to the human eye but animal visual systems in general.  \cite{MasakazuMatsugu2003}  

The main goal of a CNN is to "see" structures in images.  
These can be geometric (line, square, circle, etc.),  
typical human recognitions (face, smile, house, cat),  
and also totally inhuman and unintuitive structures (wiggly lines pointing to the left, three stripes ending up in a sharp point)

Enter kernels.  \\
Kernels are little matrices (rectangular tables of numbers) that go through an image and filter a certain structure out of it as they multiply their weights with the individual pixels. Because of this behavior, they are sometimes referred to as \textbf{filters}. \\

An aggregation of filters with an equal size is called a \textbf{convolution}. \\
When working with convolutions, we refer the inputs and outputs as \textbf{tensors}.\\
A tensor is, in layman's terms, a matrix with more than two dimensions. A tensor with three dimensions, which is called a \emph{tensor of rank three} in maths, can be imagined as a cube.\\
A convolution takes a tensor of variable dimensionality as an input and returns a tensor of rank $n$, where $n$ equals the number of filters in the convolution. The exact size of the input tensor is irrelevant, as the convolution reapplies its filters over the whole input.