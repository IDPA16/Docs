Another difficulty in evolving topologies lies in the way the topology is encoded in the genome. When a new connection is introduced in a network, it's often first a bit worse than before because it needs some time to adjust and show it's real potential. Traditional TWEANNs like to throw these kinds of topologies out of the gene pool preemptively, as they \emph{appear} to make the network worse.

NEAT solves this by again by using historical markings. The more markings a network shares with another, the more related it is to that other network. Based on this principle, NEAT groups similar networks into species, which share their fitness with each other. This means that weak individuals that are only marginally different from a proven concept are guaranteed to be temporarily protected in their niche. \cite{Stanley2002}
